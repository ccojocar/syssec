% IMPORTANT: add or remove (comment out) the boolean '\solutiontrue' below to
% create the solution document or the exercise document respectively.
% First we create the switch to make either the exercises or the solutions
\newif\ifsolution\solutionfalse
% To create the solution uncomment '\solutiontrue'
\solutiontrue

\documentclass[a4paper,11pt]{article}

\title{System Security \\
\ifsolution Solution: \else \fi
Trusted Computing and Software-Based Attestation
}

\date{\begin{tabular}{r@{ }l}%
Distribution: & 07.12.2015\\
Hand in: & 17.12.2015, 13:15 \\
\end{tabular}}

\include{author}


\usepackage[T1]{fontenc}
\usepackage{ae, aecompl}
\usepackage{a4wide}
\usepackage{boxedminipage}
\usepackage{url}
\usepackage{graphicx}
\usepackage{enumerate}

% Some useful commands and environments
\usepackage{framed}
\newenvironment{solution}%
{\par{\noindent\small\textit{Solution:}}\vspace{-12pt}\begin{framed}}%
{\end{framed}\par}

\begin{document}
\maketitle

\section{Static and Dynamic Root of Trust}

Consider the secure cloud service provider ``TrustedCloud'', located in Libya.
TrustedCloud offers secure computing services by enabling on all their servers
TPM version 1.2.  For computations performed on TrustedCloud, cloud customers
can verify system attestations using either a \textbf{static root of trust} or
a \textbf{dynamic root of trust}. The attestations customers receive are signed
by a key that is certified by Infineon (a reliable TPM manufacturer) to belong
to a valid TPM chip.

Does such a system provide cloud clients with secure cloud computing? Provide your solution
by answering the following points:

\begin{enumerate}[(a)]

\item Does the static root of trust system provide cloud customers with secure
cloud computing? Describe what security properties one can achieve through
static root of trust.

\ifsolution
\begin{solution}
Not, because the static root of trust measures the integrity of boot sequence and typically 
the operating system of cloud instances is virtualized. Most virtualization technologies 
do not require a boot loader on the virtual machines (e.g. Xen). Also normally the cloud 
services are long running applications which would require a restart before doing any sensitive 
operations.

The security property one can achieve through static root of trust is the integrity of the 
platform at load time. It verifies the code at initial loading time, but does no provide
protection against the exploits at run-time. Such kind of exploits at run-time can compromise 
the system state.    
\end{solution}\fi

\item Does the dynamic root of trust system provide cloud customers with secure
cloud computing? Describe what security properties one can achieve through
dynamic root of trust.

\ifsolution\begin{solution}
The dynamic root of trust provides more security guaranties for cloud customers but it
is not entirely secure. It eliminates the BIOS/Bootloader from the chain of trust when 
launching a Virtual Machine or a sensitive operation on a untrusted system. 

Using dynamic root of trust, remote attestation can be done on the status of the 
Virtual Machines. The Intel Trusted Execution Technology guarantees that a Virtual 
Machine accessing a confident information is unmodified.  
\end{solution}\fi

\item If you believe that the static or dynamic root of trust system is not
secure, briefly describe an attack that TrustedCloud could mount.

\ifsolution\begin{solution}
In case of static root of trust, the TrustedCloud could mount a \emph{reset attack} by trying to zero out
the PCRs without resetting the machine. Typically the PCRs resides on a \emph{Low Pin Count (LPC)} bus. The
LPC bus supports a ground driven reset line. This means when this particular line on the bus is driven to ground 
every device on this bus is supposed to reset. 

Also the dynamic root of trust can be attacked. A known attack of Intel's TXT  modifies the \emph{SMM (System Management Mode) 
Memory} before late lunched. The SMM has access to whole physical memory and is the most privileged mode on the CPU. 
The SMM Memory is protected by the chipset, but however on some chipsets bugs allow to bypass those protections. 
In this way one can execute arbitrary code in the context of the ``late launched'' Virtual Machines. 
\end{solution}\fi

\end{enumerate}

\section{\texttt{BIOS} compromise}

Assume Malta passes a law where all wireless access routers
have to be equipped with a TPM. A manufacturer of cheap hardware, D-Linksys,
realizes that there is a critical bug in the BIOS of their router.

\begin{enumerate}[(a)]
\item Explain whether the TPM can help circumvent the bug. Your solution must explain
\underline{either} \textbf{why} the TPM \underline{cannot} help circumvent the
bug, or \textbf{how} the TPM \underline{can} help.

\ifsolution\begin{solution}
The TPM can monitor the boot process by measuring the integrity of BIOS code 
executed during the startup. Also it allows to D-Linksys to verify (attest) remotely 
the platform configuration. These measures might protect against some exploits
which alter the code or configuration of BIOS, but in general do no provide 
any security against software vulnerabilities resulted from design or coding errors.   
\end{solution}\fi

\item Eventually, D-Linksys decides to update all BIOSes of their hardware. Rather
than recalling all products, the update procedure is carried online as follows:

\begin{itemize}

\item For configuration purposes, the wireless router runs a web server with
SSL.  The administrator user connects to the web server, authenticates it with
SSL, and clicks on a button to update the BIOS. Note that the web server is not
accessible outside the private network.

\item Once the update button has been pressed, the wireless access router
connects, via HTTP, to the manufacturer's web server {\tt update.d-linksys.com}
and downloads the file {\tt BIOS.COM}.

\item The wireless router re-flashes the BIOS with  {\tt BIOS.COM}, and reboots.
The update procedure is, at this point, complete.

\end{itemize}

No other steps are being carried out. Is the update procedure secure? Justify
your answer, by either arguing for its security, or by giving an example of
an attack against it.

\end{enumerate}

\ifsolution\begin{solution}
This update procedure is not secure. The file {\tt BIOS.COM} can be modified 
in transit by an attacker because is transferred in clear via HTTP. As a result, 
the router is going to be re-flashed with a malicious version of BIOS. In this case 
a TPM would help to verify the integrity of {\tt BIOS.COM} file and would allow to 
D-Linksys to attest remotely that the correct version of BIOS was upgraded.      
\end{solution}\fi

\section{Verifiable code execution via software-based attestation}


Consider the following steps of verifiable code execution, from the Pioneer paper
discussed in class:

\begin{enumerate}
\item Initialization function, receive nonce from verifier
\item Checksum code
\item Post processing
\item Send result to verifier
\item Compute hash function
\item Send result to verifier
\item Execute
\end{enumerate}

Reason about the following:

\begin{enumerate}[(a)]

\item Which parts 1-7 need to be included in the checksum computation and why?
Which parts in the hash computation and why?
\ifsolution\begin{solution}
The checksum code computes a checksum over the entire verification function 
which consists of parts 1, 2, 3, 4 and 5. It sets up an execution environment in which 
the send function, the hash function and the executable are guaranteed to run untampered 
by any malicious software on the untrusted platform.

The part 7, execute, is included in hash function computation. The hash function is
used to perform integrity measurement of the executable.  
\end{solution}\fi

\item The Pioneer paper states that the code to replace exception handlers
cannot be placed in the checksum loop, and also not in the initialization code,
because the attacker could simply skip these instructions. However, since the
verification function is computed over the checksum initializaton, loop, and
epilog code, how can you explain that the attacker can skip these exception
handler replacement instructions without the result being visible in the
checksum?

\ifsolution\begin{solution}
The attacker can skip the instructions which replace the exception handlers in the
initialization code by setting an execution break-point in the code. The processor
will generate a debug exception when trying to execute the code. In this way, the 
adversary can control the execution flow of the program.

The instructions replacing the exception handlers (e.g \emph{iret} on x86) in the checksum loop 
do not affect the value of the checksum. For this reason, the adversary can simply remove these
instructions and still compute the correct checksum within the expected time. 
\end{solution}\fi

\item The paper states, however, that the handler replacement instructions can
be placed in the epilog code. In this setting, consider an attacker sets a
debug breakpoint between steps 2 and 3, so right after the checksum is
computed, the attacker gains control. Thus, it seems the attacker can skip the
instructions as before.  What could you do about it?

\ifsolution\begin{solution}
A break point right after the checksum computation, it will increase the time within the verifier receives the checksum. The verifier detects such a situation by measuring the time elapsed from sending the nonce until receiving the checksum. If this time exceeds a certain threshold, the verifier skips the checksum verification and exits with a failure.  
\end{solution}\fi

\item The verifier wants to periodically perform verifications. What security
properties do checksum verification requests need to satisfy? How could you
perform this efficiently?

\ifsolution\begin{solution}
The security verification requests need to satisfy the checksum \emph{freshness}. The checksum must depend on a unpredictable \emph{challenge} sent by the verifier. This measure prevents that the adversary pre-computes the checksum before making changes to the verification function, and replies old checksum values. This is achieved in two ways. First, the checksum code uses the challenge to seed a Pseudo-Random Number Generator that generates input for computing the checksum. Second, the challenge is also used to initialise the checksum variable to deterministic yet unpredictable value.
\end{solution}\fi

\end{enumerate}

\end{document}
